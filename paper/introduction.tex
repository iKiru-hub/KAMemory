\section{Introduction}


The predominant description of BTSP consists of a two-stage process, where the first is the induction of a sub-threshold instructive signal \textit{IS} in the CA1 pyramidal neurons, and the second is overlapping with a supre-threshold eligibility trace \textit{ET}.
Consistent observations agree on the entorhinal cortex (EC) to be the source of the instructive signal \textit{IS}.
More specifically for the case of BTSP, the entorhinal afferences are observed to originate in layer III of the lateral region, and target the pyramidal layer in the stratum lacunosum modelculare of the distal CA1 through the temporo-ammonic pathway (Ito, 2012; Soltesz, 2018).
The action of these afferences is in the form of dendritic plateau potentials, which consist of calcium spikes evoked by sub-threshold EPSPs in the CA1 apical dendrites (Golding, 1999).
The information carried by the \textit{IS} is thought to be related to non-spatial elements of the environment, such as the occurrence of rewards or other behaviourally relevant events.
For what concerns the eligibility trace \textit{ET}, it has been identified with the projections from region CA3 of the hippocampus, gated through the Schaffer collaterals (Soltesz, 2018).
The \textit{ET} is thought to carry spatial information, such as the perceived location of the animal in the environment. This signal is generated by the CA3 pyramidal neuron, which als receive upstream input from the entorhinal cortex, but predominantly from the medial region (MEC) through the
perforant path [\textbf{cite}]. 
Its action in the context of BTSP is a supra-threshold dendritric depolarization. When this occurs, the generated synaptic trace gets integrated with the plateau potentials from the \textit{IS} to produce a long-term potentiation (LTP) of the synapse (Bittner, 2017; Milstein, 2021).
Importantly, this process is independent of the post-synaptic activity, and is meant to capture the temporal contiguity of the pre-synaptic \textit{IS} and \textit{ET} signals. The duration is typically measured in seconds, supporting the idea of consolidating information related to behaviour.
